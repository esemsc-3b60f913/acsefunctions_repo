%% Generated by Sphinx.
\def\sphinxdocclass{report}
\documentclass[a4paper,10pt,english]{sphinxmanual}
\ifdefined\pdfpxdimen
   \let\sphinxpxdimen\pdfpxdimen\else\newdimen\sphinxpxdimen
\fi \sphinxpxdimen=.75bp\relax
\ifdefined\pdfimageresolution
    \pdfimageresolution= \numexpr \dimexpr1in\relax/\sphinxpxdimen\relax
\fi
%% let collapsible pdf bookmarks panel have high depth per default
\PassOptionsToPackage{bookmarksdepth=5}{hyperref}

\PassOptionsToPackage{booktabs}{sphinx}
\PassOptionsToPackage{colorrows}{sphinx}

\PassOptionsToPackage{warn}{textcomp}
\usepackage[utf8]{inputenc}
\ifdefined\DeclareUnicodeCharacter
% support both utf8 and utf8x syntaxes
  \ifdefined\DeclareUnicodeCharacterAsOptional
    \def\sphinxDUC#1{\DeclareUnicodeCharacter{"#1}}
  \else
    \let\sphinxDUC\DeclareUnicodeCharacter
  \fi
  \sphinxDUC{00A0}{\nobreakspace}
  \sphinxDUC{2500}{\sphinxunichar{2500}}
  \sphinxDUC{2502}{\sphinxunichar{2502}}
  \sphinxDUC{2514}{\sphinxunichar{2514}}
  \sphinxDUC{251C}{\sphinxunichar{251C}}
  \sphinxDUC{2572}{\textbackslash}
\fi
\usepackage{cmap}
\usepackage[T1]{fontenc}
\usepackage{amsmath,amssymb,amstext}
\usepackage{babel}



\usepackage{tgtermes}
\usepackage{tgheros}
\renewcommand{\ttdefault}{txtt}



\usepackage[Bjarne]{fncychap}
\usepackage{sphinx}

\fvset{fontsize=auto}
\usepackage{geometry}


% Include hyperref last.
\usepackage{hyperref}
% Fix anchor placement for figures with captions.
\usepackage{hypcap}% it must be loaded after hyperref.
% Set up styles of URL: it should be placed after hyperref.
\urlstyle{same}


\usepackage{sphinxmessages}
\setcounter{tocdepth}{1}



\title{acsefunctions}
\date{Apr 18, 2025}
\release{0.1}
\author{Your Name}
\newcommand{\sphinxlogo}{\vbox{}}
\renewcommand{\releasename}{Release}
\makeindex
\begin{document}

\ifdefined\shorthandoff
  \ifnum\catcode`\=\string=\active\shorthandoff{=}\fi
  \ifnum\catcode`\"=\active\shorthandoff{"}\fi
\fi

\pagestyle{empty}
\sphinxmaketitle
\pagestyle{plain}
\sphinxtableofcontents
\pagestyle{normal}
\phantomsection\label{\detokenize{index::doc}}

\index{module@\spxentry{module}!acsefunctions.transcendental@\spxentry{acsefunctions.transcendental}}\index{acsefunctions.transcendental@\spxentry{acsefunctions.transcendental}!module@\spxentry{module}}\index{module@\spxentry{module}!acsefunctions.special@\spxentry{acsefunctions.special}}\index{acsefunctions.special@\spxentry{acsefunctions.special}!module@\spxentry{module}}\phantomsection\label{\detokenize{index:module-acsefunctions.transcendental}}\phantomsection\label{\detokenize{index:module-acsefunctions.special}}
\sphinxAtStartPar
Special functions: factorial, gamma, and Bessel.
\index{bessel() (in module acsefunctions.special)@\spxentry{bessel()}\spxextra{in module acsefunctions.special}}

\begin{fulllineitems}
\phantomsection\label{\detokenize{index:acsefunctions.special.bessel}}
\pysigstartsignatures
\pysiglinewithargsret
{\sphinxcode{\sphinxupquote{acsefunctions.special.}}\sphinxbfcode{\sphinxupquote{bessel}}}
{\sphinxparam{\DUrole{n}{alpha}}\sphinxparamcomma \sphinxparam{\DUrole{n}{x}}\sphinxparamcomma \sphinxparam{\DUrole{n}{n\_terms}\DUrole{o}{=}\DUrole{default_value}{20}}}
{}
\pysigstopsignatures
\sphinxAtStartPar
Compute the Bessel function J\_alpha(x) using its series expansion.
\begin{quote}\begin{description}
\sphinxlineitem{Parameters}\begin{itemize}
\item {} 
\sphinxAtStartPar
\sphinxstyleliteralstrong{\sphinxupquote{alpha}} (\sphinxstyleliteralemphasis{\sphinxupquote{float}}) \textendash{} Order of the Bessel function.

\item {} 
\sphinxAtStartPar
\sphinxstyleliteralstrong{\sphinxupquote{x}} (\sphinxstyleliteralemphasis{\sphinxupquote{float}}\sphinxstyleliteralemphasis{\sphinxupquote{ or }}\sphinxstyleliteralemphasis{\sphinxupquote{numpy.ndarray}}) \textendash{} Input value(s).

\item {} 
\sphinxAtStartPar
\sphinxstyleliteralstrong{\sphinxupquote{n\_terms}} (\sphinxstyleliteralemphasis{\sphinxupquote{int}}\sphinxstyleliteralemphasis{\sphinxupquote{, }}\sphinxstyleliteralemphasis{\sphinxupquote{optional}}) \textendash{} Number of terms in the series (default is 20).

\end{itemize}

\sphinxlineitem{Returns}
\sphinxAtStartPar
Computed J\_alpha(x).

\sphinxlineitem{Return type}
\sphinxAtStartPar
float or numpy.ndarray

\end{description}\end{quote}
\subsubsection*{Examples}

\begin{sphinxVerbatim}[commandchars=\\\{\}]
\PYG{g+gp}{\PYGZgt{}\PYGZgt{}\PYGZgt{} }\PYG{n}{bessel}\PYG{p}{(}\PYG{l+m+mi}{0}\PYG{p}{,} \PYG{l+m+mi}{0}\PYG{p}{)}
\PYG{g+go}{1.0}
\PYG{g+gp}{\PYGZgt{}\PYGZgt{}\PYGZgt{} }\PYG{n}{bessel}\PYG{p}{(}\PYG{l+m+mi}{0}\PYG{p}{,} \PYG{l+m+mi}{1}\PYG{p}{)}  \PYG{c+c1}{\PYGZsh{} Approximate value}
\PYG{g+go}{0.7651976865579666}
\PYG{g+gp}{\PYGZgt{}\PYGZgt{}\PYGZgt{} }\PYG{n}{bessel}\PYG{p}{(}\PYG{l+m+mi}{0}\PYG{p}{,} \PYG{n}{np}\PYG{o}{.}\PYG{n}{array}\PYG{p}{(}\PYG{p}{[}\PYG{l+m+mi}{0}\PYG{p}{,} \PYG{l+m+mi}{1}\PYG{p}{]}\PYG{p}{)}\PYG{p}{)}
\PYG{g+go}{array([1.        , 0.76519769])}
\end{sphinxVerbatim}

\end{fulllineitems}

\index{factorial() (in module acsefunctions.special)@\spxentry{factorial()}\spxextra{in module acsefunctions.special}}

\begin{fulllineitems}
\phantomsection\label{\detokenize{index:acsefunctions.special.factorial}}
\pysigstartsignatures
\pysiglinewithargsret
{\sphinxcode{\sphinxupquote{acsefunctions.special.}}\sphinxbfcode{\sphinxupquote{factorial}}}
{\sphinxparam{\DUrole{n}{n}}}
{}
\pysigstopsignatures
\sphinxAtStartPar
Compute the factorial n! for non\sphinxhyphen{}negative integers.
\begin{quote}\begin{description}
\sphinxlineitem{Parameters}
\sphinxAtStartPar
\sphinxstyleliteralstrong{\sphinxupquote{n}} (\sphinxstyleliteralemphasis{\sphinxupquote{int}}\sphinxstyleliteralemphasis{\sphinxupquote{ or }}\sphinxstyleliteralemphasis{\sphinxupquote{numpy.ndarray}}) \textendash{} Non\sphinxhyphen{}negative integer input(s).

\sphinxlineitem{Returns}
\sphinxAtStartPar
Computed n!.

\sphinxlineitem{Return type}
\sphinxAtStartPar
int or numpy.ndarray

\sphinxlineitem{Raises}
\sphinxAtStartPar
\sphinxstyleliteralstrong{\sphinxupquote{ValueError}} \textendash{} If n is negative.

\end{description}\end{quote}
\subsubsection*{Examples}

\begin{sphinxVerbatim}[commandchars=\\\{\}]
\PYG{g+gp}{\PYGZgt{}\PYGZgt{}\PYGZgt{} }\PYG{n}{factorial}\PYG{p}{(}\PYG{l+m+mi}{0}\PYG{p}{)}
\PYG{g+go}{1}
\PYG{g+gp}{\PYGZgt{}\PYGZgt{}\PYGZgt{} }\PYG{n}{factorial}\PYG{p}{(}\PYG{l+m+mi}{5}\PYG{p}{)}
\PYG{g+go}{120}
\PYG{g+gp}{\PYGZgt{}\PYGZgt{}\PYGZgt{} }\PYG{n}{factorial}\PYG{p}{(}\PYG{n}{np}\PYG{o}{.}\PYG{n}{array}\PYG{p}{(}\PYG{p}{[}\PYG{l+m+mi}{0}\PYG{p}{,} \PYG{l+m+mi}{1}\PYG{p}{,} \PYG{l+m+mi}{2}\PYG{p}{]}\PYG{p}{)}\PYG{p}{)}
\PYG{g+go}{array([1, 1, 2])}
\end{sphinxVerbatim}

\end{fulllineitems}

\index{gamma() (in module acsefunctions.special)@\spxentry{gamma()}\spxextra{in module acsefunctions.special}}

\begin{fulllineitems}
\phantomsection\label{\detokenize{index:acsefunctions.special.gamma}}
\pysigstartsignatures
\pysiglinewithargsret
{\sphinxcode{\sphinxupquote{acsefunctions.special.}}\sphinxbfcode{\sphinxupquote{gamma}}}
{\sphinxparam{\DUrole{n}{z}}\sphinxparamcomma \sphinxparam{\DUrole{n}{T}\DUrole{o}{=}\DUrole{default_value}{100}}\sphinxparamcomma \sphinxparam{\DUrole{n}{M}\DUrole{o}{=}\DUrole{default_value}{1000}}}
{}
\pysigstopsignatures
\sphinxAtStartPar
Compute the gamma function gamma(z) for z \textgreater{} 0 using numerical integration.

\sphinxAtStartPar
Uses trapezoidal rule on gamma(z) = ∫\_0\textasciicircum{}\(\infty\) t\textasciicircum{}(z\sphinxhyphen{}1) e\textasciicircum{}(\sphinxhyphen{}t) dt.
\begin{quote}\begin{description}
\sphinxlineitem{Parameters}\begin{itemize}
\item {} 
\sphinxAtStartPar
\sphinxstyleliteralstrong{\sphinxupquote{z}} (\sphinxstyleliteralemphasis{\sphinxupquote{float}}\sphinxstyleliteralemphasis{\sphinxupquote{ or }}\sphinxstyleliteralemphasis{\sphinxupquote{numpy.ndarray}}) \textendash{} Input value(s), must be positive.

\item {} 
\sphinxAtStartPar
\sphinxstyleliteralstrong{\sphinxupquote{T}} (\sphinxstyleliteralemphasis{\sphinxupquote{float}}\sphinxstyleliteralemphasis{\sphinxupquote{, }}\sphinxstyleliteralemphasis{\sphinxupquote{optional}}) \textendash{} Upper integration limit (default is 100).

\item {} 
\sphinxAtStartPar
\sphinxstyleliteralstrong{\sphinxupquote{M}} (\sphinxstyleliteralemphasis{\sphinxupquote{int}}\sphinxstyleliteralemphasis{\sphinxupquote{, }}\sphinxstyleliteralemphasis{\sphinxupquote{optional}}) \textendash{} Number of integration points (default is 1000).

\end{itemize}

\sphinxlineitem{Returns}
\sphinxAtStartPar
Computed gamma(z).

\sphinxlineitem{Return type}
\sphinxAtStartPar
float or numpy.ndarray

\sphinxlineitem{Raises}
\sphinxAtStartPar
\sphinxstyleliteralstrong{\sphinxupquote{ValueError}} \textendash{} If z \textless{}= 0.

\end{description}\end{quote}
\subsubsection*{Examples}

\begin{sphinxVerbatim}[commandchars=\\\{\}]
\PYG{g+gp}{\PYGZgt{}\PYGZgt{}\PYGZgt{} }\PYG{n}{gamma}\PYG{p}{(}\PYG{l+m+mi}{1}\PYG{p}{)}
\PYG{g+go}{1.0}
\PYG{g+gp}{\PYGZgt{}\PYGZgt{}\PYGZgt{} }\PYG{n}{gamma}\PYG{p}{(}\PYG{l+m+mf}{0.5}\PYG{p}{)}  \PYG{c+c1}{\PYGZsh{} Approximately sqrt(pi)}
\PYG{g+go}{1.7724538209055159}
\PYG{g+gp}{\PYGZgt{}\PYGZgt{}\PYGZgt{} }\PYG{n}{gamma}\PYG{p}{(}\PYG{n}{np}\PYG{o}{.}\PYG{n}{array}\PYG{p}{(}\PYG{p}{[}\PYG{l+m+mi}{1}\PYG{p}{,} \PYG{l+m+mi}{2}\PYG{p}{]}\PYG{p}{)}\PYG{p}{)}
\PYG{g+go}{array([1., 1.])}
\end{sphinxVerbatim}

\end{fulllineitems}



\renewcommand{\indexname}{Python Module Index}
\begin{sphinxtheindex}
\let\bigletter\sphinxstyleindexlettergroup
\bigletter{a}
\item\relax\sphinxstyleindexentry{acsefunctions.special}\sphinxstyleindexpageref{index:\detokenize{module-acsefunctions.special}}
\item\relax\sphinxstyleindexentry{acsefunctions.transcendental}\sphinxstyleindexpageref{index:\detokenize{module-acsefunctions.transcendental}}
\end{sphinxtheindex}

\renewcommand{\indexname}{Index}
\printindex
\end{document}